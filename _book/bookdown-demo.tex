\documentclass[]{book}
\usepackage{lmodern}
\usepackage{amssymb,amsmath}
\usepackage{ifxetex,ifluatex}
\usepackage{fixltx2e} % provides \textsubscript
\ifnum 0\ifxetex 1\fi\ifluatex 1\fi=0 % if pdftex
  \usepackage[T1]{fontenc}
  \usepackage[utf8]{inputenc}
\else % if luatex or xelatex
  \ifxetex
    \usepackage{mathspec}
  \else
    \usepackage{fontspec}
  \fi
  \defaultfontfeatures{Ligatures=TeX,Scale=MatchLowercase}
\fi
% use upquote if available, for straight quotes in verbatim environments
\IfFileExists{upquote.sty}{\usepackage{upquote}}{}
% use microtype if available
\IfFileExists{microtype.sty}{%
\usepackage{microtype}
\UseMicrotypeSet[protrusion]{basicmath} % disable protrusion for tt fonts
}{}
\usepackage[margin=1in]{geometry}
\usepackage{hyperref}
\hypersetup{unicode=true,
            pdftitle={Elaborando um Compêndio de Pesquisa em RMarkdown},
            pdfauthor={Emerson M. Del Ponte},
            pdfborder={0 0 0},
            breaklinks=true}
\urlstyle{same}  % don't use monospace font for urls
\usepackage{natbib}
\bibliographystyle{apalike}
\usepackage{longtable,booktabs}
\usepackage{graphicx,grffile}
\makeatletter
\def\maxwidth{\ifdim\Gin@nat@width>\linewidth\linewidth\else\Gin@nat@width\fi}
\def\maxheight{\ifdim\Gin@nat@height>\textheight\textheight\else\Gin@nat@height\fi}
\makeatother
% Scale images if necessary, so that they will not overflow the page
% margins by default, and it is still possible to overwrite the defaults
% using explicit options in \includegraphics[width, height, ...]{}
\setkeys{Gin}{width=\maxwidth,height=\maxheight,keepaspectratio}
\IfFileExists{parskip.sty}{%
\usepackage{parskip}
}{% else
\setlength{\parindent}{0pt}
\setlength{\parskip}{6pt plus 2pt minus 1pt}
}
\setlength{\emergencystretch}{3em}  % prevent overfull lines
\providecommand{\tightlist}{%
  \setlength{\itemsep}{0pt}\setlength{\parskip}{0pt}}
\setcounter{secnumdepth}{5}
% Redefines (sub)paragraphs to behave more like sections
\ifx\paragraph\undefined\else
\let\oldparagraph\paragraph
\renewcommand{\paragraph}[1]{\oldparagraph{#1}\mbox{}}
\fi
\ifx\subparagraph\undefined\else
\let\oldsubparagraph\subparagraph
\renewcommand{\subparagraph}[1]{\oldsubparagraph{#1}\mbox{}}
\fi

%%% Use protect on footnotes to avoid problems with footnotes in titles
\let\rmarkdownfootnote\footnote%
\def\footnote{\protect\rmarkdownfootnote}

%%% Change title format to be more compact
\usepackage{titling}

% Create subtitle command for use in maketitle
\newcommand{\subtitle}[1]{
  \posttitle{
    \begin{center}\large#1\end{center}
    }
}

\setlength{\droptitle}{-2em}
  \title{Elaborando um Compêndio de Pesquisa em RMarkdown}
  \pretitle{\vspace{\droptitle}\centering\huge}
  \posttitle{\par}
  \author{Emerson M. Del Ponte}
  \preauthor{\centering\large\emph}
  \postauthor{\par}
  \predate{\centering\large\emph}
  \postdate{\par}
  \date{2017-11-15}

\usepackage{booktabs}
\usepackage{amsthm}
\makeatletter
\def\thm@space@setup{%
  \thm@preskip=8pt plus 2pt minus 4pt
  \thm@postskip=\thm@preskip
}
\makeatother

\begin{document}
\maketitle

{
\setcounter{tocdepth}{1}
\tableofcontents
}
\hypertarget{prefacio}{%
\chapter*{Prefácio}\label{prefacio}}
\addcontentsline{toc}{chapter}{Prefácio}

\textbf{Pesquisa Reproduzível} (sin. reprodutível) é um tema da ciencia
que tem despertado muito a atenção de pesquisadores, agências de fomento
e a mídia acadêmica nos últimos anos. São frequentes os relatos de que
um estudo que foi repetido gerou resultados diferentes ou mesmo
discortantes de um estudo anterior. Os próprios pesquisadores tem se
manifestado com grande preocupação com uma alegada ``crise de
reprodutibilidade'' na ciência.

As possíveis causas e algumas soluções para minimizar o problema vem
sendo discutidas e algumas ações implementadas. É importante que o
estudante e pesquisador em geral tenha conhecimento sobre a correta
definição dos termos para melhorar a comunicação. Reprodutibilidade tem
diferentes significados dependendo do contexto. Aqui, definimos como:

\begin{quote}
Capacidade de um pesquisador em chegar ao resultado de um estudo prévio
usando os mesmos materiais (dados) e métodos (estatística) da pesquisa
original.
\end{quote}

Portanto, é condição \emph{sine qua non} que o pesquisador independente
tenha:

\begin{itemize}
\tightlist
\item
  acesso aos dados
\item
  saiba detalhes de como a análise foi feita
\end{itemize}

Na mídia, os resultados discordantes de estudos se refere, na verdade, à
\textbf{replicabilidade}, ou a \textbf{reprodutibilidade inferencial},
segundo alguns autores. Em ciência, parte-se do princípio de que os
resultados de uma pesquisa publicados em revistas com corpo editorial
tenham sido obtidos segundo os princípios que regem os métodos e a ética
científica. No entanto, os editores e revisores, quase sempre, não tem
como verificar se todos os passos do trabalho, especialmente a análise
dos dados, foram executados corretamente, uma vez que avaliam apenas um
produto - o artigo científico.

Um artigo científico é escrito e submetido para publicação segundo
convenções da academia que definem o conteúdo mínimo para que o trabalho
seja avaliado pelos pares. Normalmente esse conteúdo consiste no texto,
gráficos e tabelas e, idealmente, um material suplementar (o que ainda
parece ser desconhecido para muitos pesquisadores).

\hypertarget{ambiente-computacional}{%
\section*{Ambiente computacional}\label{ambiente-computacional}}
\addcontentsline{toc}{section}{Ambiente computacional}

Um cientista que objetiva que sua pesquisa seja reproduzida por outros
grupos deve se preocupar em disponibilizar também os dados e os
procedimentos de análise para que possam ser inspecionados e
reutilizados. Esse trabalho também exige procedimentos padronizados de
conteúdo e formato de arquivos (dados, relatórios, imagens, etc.) os
quais devem ser comentados de maneira clara para que outros
pesquisadores possam reproduzir, melhorar ou expandir uma análise
estatística, reutilizar os dados em outro trabalho ou combinar com
outros conjuntos. Uma condição para que isso ocorra é que os
pesquisadores usem um mesmo ambiente computacional que seja
preferenciamente de código aberto. Dentre os ambientes, nossa
preferência é pelo R e conjunto de ferramentas integradas no ambiente
RStudio.

Nesse livro, o estudante ou pesquisador interessado aprenderá as boas
práticas de pesquisa reproduzível que devem ser incorporadas na rotina
de trabalho. Com hábitos simples, persistência, sistemática e um pouco
de dedicação na linguagem R, o pesquisador moderno estará adotando
práticas que contribuirão para a maior transparência na ciência.

\hypertarget{slides-do-autor}{%
\section*{Slides do autor}\label{slides-do-autor}}
\addcontentsline{toc}{section}{Slides do autor}

A primeira versão deste material foi elaborada para um minicurso de 8h
oferecido aos estudantes do Programa de Pós-graduação em Fitopatologia
dia 13 de novembro de 2017. Os slides da apresentação podem ser
visualizado abaixo.

\hypertarget{agradecimentos}{%
\section*{Agradecimentos}\label{agradecimentos}}
\addcontentsline{toc}{section}{Agradecimentos}

\hypertarget{intro}{%
\chapter{Requisitos básicos}\label{intro}}

\hypertarget{uma-nova-rotina}{%
\section{Uma nova rotina}\label{uma-nova-rotina}}

Para um estudante ou cientista que está iniciando um projeto é
importante que as boas práticas de PR sejam incorporadas no seu dia a
dia, e que sejam implementadas desde a concepção e o planejamento do
mesmo.

São atividades que dependem essencialmente de uma grande capacidade
organizacional e admistrativa de tempo e esforço no planejamento,
condução e documentação de tudo que é feito. É preciso seguir rotinas e
gerar documentos que seguem certas normas de padronização, especialmente
se o trabalho é feito de forma colaborativa. Analogamente, é como
escrever e formatar um artigo científico que deve ser estruturado e
apresentado segundo determinadas normas. Aqui, o produto gerado não é
somente o documento do manuscrito e um punhado de gráficos, mas sim tudo
que foi gerado durante a pesquisa e que precisa estar bem organizado e
formatado para uso posterior e publicação/divulgação. Para obter sucesso
na implementação de uma PR, é preciso:

\begin{itemize}
\tightlist
\item
  Ser diligente e sistemático
\item
  Aprender novas ferramentas (computacionais)
\item
  Aprender a organizar arquivos diversos
\item
  Documentar todas as etapas do trabalho
\end{itemize}

No dia a dia, os pesquisadores não sobrevivem se os computadores como
ferramenta central de trabalho. Atualmente, não é preciso ser um
``nerd'' para que se possa utilizar com bastante eficiência os
computadores que estão hoje cada vez mais portáteis e de fácil uso, para
ser eficiente e produtivo no trabalho. Em algumas áreas da pesquisa é
necessário maior envolvimento com linguagens de programação, programas
específicos que exigem um esforço de aprendizado.

No entanto, na PR o mais importante e desafior é certamente aprender a
sistemática de trabalho do que ser um expert em programação - mas é
necessário sim aprender alguma linguagem de programação (R ou Python)
para implementar as práticas de PR. Durante nossa formação acadêmica não
recebemos nenhum ou muito pouco treinamento em como preparar e organizar
de maneira apropriadas os arquivos diversos incluindo dados, códigos,
gráficos, tabelas, manuscrito, figuras, etc.

Apender uma rotina de PR é fundamental para:

\begin{enumerate}
\def\labelenumi{\arabic{enumi})}
\tightlist
\item
  Facilitar o nosso próprio trabalho de análise-reanálise
\item
  Permitir o uso dos dados e códigos por outras pessoas (seu
  orientador!)
\item
  Compartilhar a ``pipeline'' da análise, ou seja, explicar o que, por
  que e como foi feito
\end{enumerate}

Quando não somos treinados a trabalhar seguindo as boas práticas de PR,
é muito comum: criar um número grande arquivos e versões desnecessárias
que dificultam o processo; gerar inconsistência e redundância nas
análises; não ter um controle adequado de versões e dificuldade quando é
solicitado o compartilhamento do trabalhos - ou seja, levará um tempo
grande só para organizar a ``bagunça'' que foi gerada durante o processo
e que só o próprio pesquisador entende, quando entende! Práticas que
deveriam ser simples como refazer um gráfico ou estatísticas após
receber os pareceres de revisores se tornam um verdadeiro pesadelo para
alguns pesquisadores, o que contribui para o atraso na publicação de
artigos.

\hypertarget{ferramentas-computacionais}{%
\section{Ferramentas computacionais}\label{ferramentas-computacionais}}

Segundo \href{https://yihui.name}{Yihui Xie}, um dos principais
desenvolvedores do R da empresa RStudio de programas (ex. knitr,
rMarkdown, bookdown, etc) que visam facilitar a pesquisa reproduzível:

\begin{quote}
The final product of research is not only the paper itself, but also the
full computation environment used to produce the results in the paper
such as the code and data necessary for reproduction of the results and
building upon the research (Xie et al.~2014).
\end{quote}

Dentre os ambientes de programação disponíveis, as ferramentas mais
usadas para implementar uma PR de maneira efetiva (dados, análises e
saídas são combinados, idealmente, em um único ambiente de programação),
são baseados em duas linguagens principais: Python e R, cujos produtos
principais são Jupyter Notebooks e RMarkdown, respectivamente. Esses
pacotes ou rotinas facilitam sobremaneira a documentação e reprodução
das análises bem como aceleram a obtenção dos resultados e visualizações
assim que novos dados forem adicionados ou reanálises são necessárias.

Além de aprender a utilizar esses programas, é importante que o
pesquisador aprenda como usar efetivamente planilhas eletrônicas para
reunir e organizar os dados que serão usados na pesquisa. Por
princípios, as planilhas eletrônicas como Excel, Libre Office Calc,
Numbers e Google Sheets são usadas apenas para armazenar os dados e não
para processar, transformar, visualizar ou fazer sumários prévios. O
motivo é muito simples: esses procedimentos todos feitos com movimentos
de mouse não são reproduzíveis! além disso, na PR os dados originais
levantados ou recebidos devem ser mantidos na sua forma original. Caso
seja modificado de forma que é mais fácil fazer em uma planilha como
renomear variáveis, é importante manter sempre uma planilha não
manipulada como referência.

\hypertarget{compedio-de-pesquisa}{%
\chapter{Compêdio de pesquisa}\label{compedio-de-pesquisa}}

\hypertarget{definicao}{%
\section{Definição}\label{definicao}}

Um compêndio (do inglês, compendium) pode ser
\href{https://www.priberam.pt/dlpo/compêndio}{definido} como uma
compilação em que se encontra resumido o mais indispensável de um
estudo. O termo ``research compendium'' foi cunhado em meados dos anos
2000 em uma publicação sobre pesquisa reproduzível com enfoque em
análise estatística.

\begin{quote}
We introduce the concept of a compendium as both a container for the
different elements that make up the document and its computations
(i.e.~text, code, data, \ldots{}), and as a means for distributing,
managing and updating the
collection.\citep{Robert_Gentleman_Department_of_Biostatistics_Harvard_University2004-qt}
\end{quote}

Na visão dos autores, o compêndio tem como base um ou mais documentos
dinâmicos, a partir dos quais são gerados os demais documentos estáticos
como um PDF a partir de um arquivo TEX (há mais de 10 anos o TEX era o
equivalente ao Markdown). Também, esses documentos dinâmicos (e seus
elementos de texto, código e dados) devem ser facilmente extraídos e
processados de diferentes maneiras pelo autor e pelos leitores.

Portanto, é importante que um compêndio da pesquisa seja desenvolvido de
maneira padronizada para que outros pesquisadores, de forma facilitada,
possam inspecionar, reproduzir e ampliar a pesquisa. Os princípios
básicos para a confecção de um compêndio de pesquisa são:

\begin{enumerate}
\def\labelenumi{\arabic{enumi})}
\tightlist
\item
  Organização segundo a convenção criada pela academia
\item
  Separação clara de dados, códigos e saídas
\item
  Especificação do ambiente computacional
\item
  Documentação detalhada de cada elemento e rotina
\end{enumerate}

Um dos objetivo de se criar um compêndio de pesquisa é, em primeira
instância, facilitar o trabalho do pesquisador e permitir que ele
distribua os produtos gerados pela pesquisa de uma maneira mais ampla
possível, visando a dar visibilidade ao trabalho. Além disso, a
eficiência do trabalho é aumentada com a incorporação de uma sistemática
que pode ser replicada em outros projetos, dimuindo assim os custos
operacionais e acelerando o trabalho. Tem se discutido que publicações
que são acompanhas de um compêndio tendem a receber maior atenção,
credibilidade e citações uma vez que o compendio é um tipo de publicação
por si só. Imagine o quanto pode ser útil tornar o compêndio público
antes da submissão do trabalho de forma que editores e revisores possam
revisar os dados e os métodos (estatísticos principalmente) que foram
utilizados, entender as decisões tomadas no processo analítico e os
resultados obticos. Alem disso, uma vez que o compêndio é publicado, os
autores poderão receber comentários dos pares e dos revisores para
melhorar o trabalho como um todo.

Antes de falarmos sobre como organizar o compêndio de pesquisa, veremos
aspectos e cuidados específicos de organização dos dados.

\hypertarget{dados}{%
\section{Dados}\label{dados}}

Segundo Wilkinson et al. (2016), os dados devem ser organizados segundo
o princípio \textbf{FAIR} = \textbf{F}indable, \textbf{A}ccessible,
\textbf{I}nteroperable and \textbf{R}eusable.

Compartilhar os dados (o que pode ser aplicado também aos códigos,
significa facilitar a distribuição e o acesso pela comunidade
científica, ou seja que eles sejam facilmente \textbf{encontrados} e
\textbf{acessados}. Qual a vantagem disso? reproduzir os resultados
originais e pertimitir que novas análises sejam feitas usando os mesmos
dados, ou mesmo combinando com outros conjuntos de dados (metanálise). É
importante que os dados estejam em um formato que seja de fácil
entendimento para facilitar o uso. Três boas práticas são recomendadas:

\begin{enumerate}
\def\labelenumi{\arabic{enumi}.}
\tightlist
\item
  Documentação: dados bem documentados e descritos (metadados) são mais
  fáceis de entender
\item
  Formataçao: dados formatados apropriadamente podem ser usados em
  diversos programas de computador
\item
  Distribuição: depósito em repositórios conhecidos e com licença aberta
  facilita que sejam encontrados e reusados
\end{enumerate}

Práticas de compartilhamento de dados talvez ainda não sejam valorizadas
pela maioria dos pesquisadores, haja visto que a maioria dos trabalhos
ainda não disponibilizam os dados originais. As vantagens óbvias seria a
reprodução e possível melhorias na análise, o reuso dos dados em
metanálises para chegar a conclusões gerais e gerar novo conhecimento
que só é possível com dados compartilhados em larga escala. Mas por que
ainda os pesquisadores não compartilham os dados? os motivos podem ser
ligados ao receio de perder uma competição por publicações ou mesmo a
falta de conhecimento sobre como fazer o compartilhamento. A percepção
que domina é que organizar e depositar dados é difícil tecnicamente ou
leva muito tempo,

\hypertarget{formato}{%
\section{Formato}\label{formato}}

\hypertarget{planilhas-eletronicas}{%
\section{Planilhas eletronicas}\label{planilhas-eletronicas}}

\hypertarget{nomes}{%
\section{Nomes}\label{nomes}}

\hypertarget{copias}{%
\section{Copias}\label{copias}}

\hypertarget{formato-1}{%
\section{Formato}\label{formato-1}}

\hypertarget{deposito}{%
\section{Depósito}\label{deposito}}

\hypertarget{licenca}{%
\section{Licença}\label{licenca}}

\bibliography{book.bib,packages.bib}


\end{document}
